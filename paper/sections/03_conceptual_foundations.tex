\section{Conceptual Foundations}

We distinguish execution structure from observation and avoid assigning meaning to either. The theory uses only structural terms.

\paragraph{Execution.}
Execution is a sequence of atomic events that can be ordered without reference to wall clock time. Execution is not defined by outcome meaning, only by the order and identity of events.

\paragraph{Order and time.}
Order is derived from sequence position. Time is not used to define order. Clock time may be recorded as observation but has no role in deterministic projection.

\paragraph{Observation and derivation.}
Observations are artifacts produced by execution. They are not derivational inputs to deterministic projections. A projection that changes when observations change is not deterministic in the sense of this theory.

\paragraph{Structural determinism.}
Determinism is defined as invariance of a canonical projection under fixed authoritative execution inputs $I_{\text{exec}}$. It is a property of structure, not of outcome meaning.

\paragraph{Normativity and derivation.}
Execution is descriptive. Normativity is an external operator not defined in this theory. Observations are explicitly excluded from derivations that define deterministic projections.
